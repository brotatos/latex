\documentclass[11pt]{article}
\usepackage{tabularx}
\usepackage[a4paper,landscape]{geometry}
\usepackage{bigstrut}

\newcommand{\ihat}{\hat{\textbf{\i}}}
\newcommand{\jhat}{\hat{\textbf{\j}}}
\newcommand{\khat}{\hat{\textbf{k}}}

\title{AP Physics C: Mechanics Outline}
\date{June 2012}
\begin{document}
\maketitle
\pagebreak

\section{Motion in One Dimension}
\begin{tabularx}{\textwidth}{l| X l}
    Name                       & Definition                                                               & Equations / Examples \\ \hline
    Position                   & location of a particle with respect to a chosen reference point (origin) & \\ \hline
    Displacement               & change in position over time interval                                    & $ \Delta x=x_{f} - x_{i} $ \bigstrut \\ \hline
    Distance                   & length of particle followed by a particle                                & \\ \hline
    Vector quantity            & requiring both direction and magnitude                                   & 5m north \\ \hline
    Scalar quantity            & requiring only magnitude                                                 & 5m \\ \hline
    Average Velocity           & displacement over time interval                                          & $ \overline{v}_{x} = \frac{ \Delta x}{ \Delta t } $ \bigstrut \\ \hline
    Average Speed              & total distance over total time to travel                                 & $ \overline{speed} = \frac{ total distance }{ total time } $ \bigstrut \\ \hline
    Instantaneous Velocity     & velocity at one time                                                     & $ v_{x} = \lim_{ \Delta t \to 0} \frac{\Delta x}{\Delta t} = \frac{dx}{dt} $ \bigstrut \\ \hline
    Instantaneous Speed        & speed at one time                                                        & \\ \hline
    Average Acceleration       & change in velocity over interval of time                                 & $ \overline{a}_{x} = \frac{\Delta v_{x}}{\Delta t} = \frac{V_{f}-V_{i}}{t_{f}-t_{i}} $ \bigstrut \\ \hline
    Instantaneous Acceleration & change in velocity at one time                                           & $ \overline{a}_{x} = \lim_{\Delta t \to 0} \frac{\Delta V_{s}}{\Delta t} = \frac{dv_{s}}{dt} $ \bigstrut \\ \hline
    Kinematics                 & 1-dimensional motion with constant acceleration                          & $ v= v_{o} + a \cdot t $ \\*                                                                                       &  & $ \Delta x = \frac{v + v_{o}}{2}t $ \\* &  & $ \Delta x = v_{o}t + .5at^{2} $ \\* &  & $ v^{2} = v^{2}_{x}+ 2\Delta x \cdot a$ \bigstrut \\ \hline
    Freely Falling Object      & any object moving under influence of gravity alone \\*                   & $ a_{x} $ is always down (regardless of initial motion) \\ \hline
\end{tabularx}

\section{Vectors}
\begin{tabularx}{\textwidth}{l| X l}
    Name                            & Definition                                & Equations / Examples \\ \hline
    Adding Vectors                  & draw from head $ \to $ tail \\ \hline
    Subtracting Vectors             & same as adding negatives of vectors       & $ \vec{A} - \vec{B} = \vec{A} + - \vec{B} $ \bigstrut \\ \hline
    Components                      & projects of vectors along coordinate axis & $ A_{x} + A_{y} = A $ \bigstrut \\ \hline
    Unit Vector                     & dimensions vectors of magnitude 1         & $ \ihat (i-hat) : x-axis $ \\* &  & $ \jhat (j-hat): y-axis $ \\* &  & $ \khat (k-hat): z-axis $ \bigstrut \\ \hline
    Positional Vector $ (\hat{r}) $ & vector lying in plane                     &
\end{tabularx}

\section{Motion in Two Dimensions}
\begin{tabularx}{\textwidth}{l| X l}
    Name                     & Definition                                                                                             & Equations / Examples \\ \hline
    Uniform Circular Motion  & object moving in circular path with constant speed $ v $                                               & \\ \hline
    Centripetal Acceleration & where the acceleration vector is always perpendicular to the path and always points towards the center & $ a_{c} $ or $  a_{r} =\frac{v^2}{r} $ \bigstrut \\ \hline
    Period                   & time required for one complete revolution                                                              & $ T = \frac{2\pi r}{v} $ \bigstrut \\ \hline
    Tangential Acceleration  & causes change in speed of particle                                                                     & $ a_{t} = \frac{d|v|}{dt} $ \\*                            &  & $ a_{r} = -a_{c} = \frac{-v^2}{r} $ \\* &  & $ a = a_{t} + a_{r} $ \bigstrut \\ \hline
    Relative Velocity        & Velocity in reference to one perspective                                                               & \\  \hline
    Relative Acceleration    & Acceleration in reference to one perspective                                                           & \\ \hline
\end{tabularx}

\section{The Laws of Motion}
\begin{tabularx}{\textwidth}{l| X l}
    Name & Definition & Equations / Examples \\ \hline
    Net Force & Vector sum of all forces acting on an object & $ \sum{F} $ \bigstrut \\ \hline
    Equlibrium & & $ \sum{F} = 0 $ \bigstrut \\ \hline
    Netwon's Laws of Motion: && \\* First & When no force acts on an object, the acceleration is 0. & \\* Second & Acceleration of an object is directly proportional to the net force acting on it and inversely proportional to its mass. & $ \sum{F} = ma $ \\* Third & Every force has an equal but opposite force (action/reaction) & $ F = -F $ \\ \hline
    Inertia & tendency of an object to resist any attempt to change its velocity & \\ \hline
    Gravitational Force & attractive force exterted by earth & $ F_{g} = mg $ \bigstrut \\ \hline
    Normal Force & the reaction force of $ F_{g} $; perpendicular to $ F_{g} $ & $ F_{g} = -F_{N} $ \bigstrut \\ \hline
    Free-Body Diagram & depicts an object with all of the forces acting on it & \\ \hline
    Tension & a force of pull on an object (mainly found in strings and ropes) & $ T $ \\ \hline
    Friction & resistance to motion either on surface or in air/water & \\ \hline
    Force of Static Friction & resistance to motion that keeps an object in place & $ f_{s} \le \mu n $  \\ \hline
    Force of Kinetic Friction & friction force for an object in motion & $ f_{k} = \mu n $ \\ \hline
    Coefficient of Static Friction & dimensionless constant & $ \mu_{s} $ \bigstrut \\ \hline
    Coefficient of Kinetic Friction & dimensionless constant (generally less than $ \mu_{s} $ & $ \mu{k} $ \bigstrut \\ \hline
\end{tabularx}

\section{Circular Motion and Other Applications of Newton's Laws}
\begin{tabularx}{\textwidth}{l| X l}
    Name & Definition & Equations / Examples \\ \hline
         & & $ \sum{F} = ma_{c} = m\frac{v^2}{r} $ \\ \hline
\end{tabularx}

\section{Energy and Energy Transfer}
\begin{tabularx}{\textwidth}{l| X l}
    Name & Definition & Equations / Examples \\ \hline
    Work & done on a system by an agent exerting a constant force & $ W \equiv F\Delta{r}cos(\theta) $ \\* & an energy transfer between a system and agent & $ W \equiv F\Delta{x} $ \\* & & Varying force: $ W = \int_{x_{i}}^{x_{f}} F_{x}dx $ \bigstrut \\ \hline 
    Power & time rate of energy transfer & $ P \equiv \frac{W}{\Delta t} = \frac{dW}{dt} = F \cdot \frac{dr}{dt} = F \cdot v $ \bigstrut \\ \hline
\end{tabularx}

\section{Potential Energy}
\begin{tabularx}{\textwidth}{l| X l}
    Name & Definition & Equations / Examples \\ \hline
    Potential Energy & energy possessed by virtue of position relative to others & $ U $ \\ \hline
    Gravitational Potential Energy & potential to be pushed around by a force & $ U_{g} \equiv mgy $ \\* & & $ W = \Delta U_{g} $ \\ \hline
    Conservation of Mechanical Energy & no energy transfer occurs &  $ E_{mech} = K + U $ \\*  (in an isolated system) &  sum of kinetic and potential energies remain constant & $ E_{f} = E_{i} $ \bigstrut \\ \hline
    Elastic Potential Energy & potential energy of springs & $ U_{s} \equiv \frac{1}{2}kx^2 $ \bigstrut \\ \hline
    Conservative Forces & force where only initial and final states matter (energy is conserved) & $ W_{c} = U_{f} - U_{i} = \Delta U $ \bigstrut \\ \hline
\end{tabularx}

\end{document}

